\section{Il modello relazionale}
Il modello attualmente più diffuso e con il quale si organizzano i database è chiamato
\textbf{modello relazionale}. Tale nome è dovuto al fatto che esso è basato sul
concetto matematico di \emph{relazione}.
\begin{defn}
 Dati $n$ domini $D_1$, $D_2$, $\ldots$, $D_n$ non necessariamente distinti, una 
 relazione $r$ su $D_1$, $D_2$, $\ldots$, $D_n$ è un sottoinsieme del prodotto cartesiano
 $D_1 \times D_2 \times \ldots \times D_n$.
\end{defn}
Ogni elemento di $r$ è una n-pla $(d_1, d_2, \ldots, d_n)$  $\forall \ d_i \in D_i$

\subsection{Relazioni e tabelle}
Una relazione $r$ può essere rappresentata mediante una tabella: le righe corrispondono
agli elementi della relazione (n-ple), le colonne corrispondono ai domini $D_1\ldots D_n$.

\begin{exmp}
La seguente tabella rappresenta una relazione con due 4-ple sui domini $String$, $String$, 
$Int$ e $Real$.
 \begin{center}
  \begin{tabular}{llll}
    String & String & Int & Real\\
    \hline
    Paolo & Rossi & 2 & 26,5\\
    Mario & Biachi & 10 & 28,7
  \end{tabular}
 \end{center}
 A questo punto è poco chiaro cosa rappresenti tale relazione. Bisogna stabilire metodo per
 fornire un'interpretazione standard che rispecchi la realtà di interesse a cui essa si 
 riferisce:a questo fine diamo nomi alle colonne e alla tabella stessa. La precedente tabella
 diverrà quindi:

\begin{center}
  \begin{tabular}{l | l | l | l}
    Nome & Cognome & Esami & Media\\
    \hline
    Paolo & Rossi & 2 & 26,5\\
    Mario & Biachi & 10 & 28,7
  \end{tabular}
 \end{center}
 Ora risulta più evidente che la realtà di interesse sono gli studenti di una data università. 
 Daremo quindi il nome \emph{Studenti} a tale tabella.
\end{exmp}

Il nome dato ad ogni colonna viene chiamato \textbf{attributo}.
\begin{defn}
 Uno \emph{schema di relazione} è un insieme di attributi. 
\end{defn}
\begin{defn}
 Uno \emph{schema di base di dati relazionale} è un insieme $\{R_1, R_2, \ldots, R_n\}$ di 
 schemi di relazione. 
\end{defn}
\begin{defn}
 Una \emph{base di dati relazionale} con schema $\{R_1, R_2, \ldots, R_n\}$ è un insieme
 $\{r_1, r_2, \ldots, r_n\}$ dove $r_i$ è \emph{un'istanza} di relazione con schema $R_i$.
\end{defn}

