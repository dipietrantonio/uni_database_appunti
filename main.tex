\documentclass[a4paper,10pt]{article}
\usepackage[utf8]{inputenc}
\usepackage[hidelinks]{hyperref}
\usepackage{ragged2e}
\usepackage{graphicx}
\usepackage{wrapfig}
\usepackage[shortlabels]{enumitem}
\usepackage{longtable}
\usepackage[ddmmyyyy]{datetime}
\usepackage{amsthm,amsbsy,multicol,amssymb,mathtools, 
	    xcolor, mdframed,lmodern,textcomp, geometry}
 \geometry{
 a4paper,
 left=30mm,
 right=30mm,
 top=23mm,
 bottom=23mm,
 }
 
%comando per i footnote
\renewcommand{\thefootnote}{\arabic{footnote}}

 %comando per gli hyperlink personalizzati
\newcommand{\linkto}[2]{\hyperref[#1]{\underline{\textsc{#2}}}}
\newcommand{\textbi}[1]{\textbf{\emph{#1}}}
%titolo del documento%
\title{\Huge Basi di Dati - Appunti}
\author{Cristian Di Pietratonio}
\renewcommand{\dateseparator}{-}
\date{\today}

%importiamo le definizioni di nuove funzioni
%definiamo i colori
\definecolor{c_theo}{HTML}{FFF99C}
\definecolor{c_exmp}{HTML}{B9FF9D}
\definecolor{c_def}{HTML}{CEDAFF}
\definecolor{c_alg}{HTML}{FF2020}
\newtheoremstyle{mys}
  {\topsep}
  {\topsep}%
  {}
  {}%
  {\bfseries}{}%
  {\newline}{}%
\theoremstyle{mys}

%riquadro della proposizione
\newmdtheoremenv[%
roundcorner = 10pt,
linecolor=lightgray!30,
backgroundcolor=lightgray!30,%
innertopmargin=3pt]{prop}{Proposizione}[section]



%riquadro del teorema
\newmdtheoremenv[%
linecolor=c_theo!55,
backgroundcolor=c_theo!55,%
innertopmargin=3pt]{theo}{Teorema}[section]

%riquadro esempio
\newmdtheoremenv[%
linecolor=c_exmp!40,
backgroundcolor=c_exmp!40,%
innertopmargin=3pt]{exmp}{Esempio}[section]

%riquadro lemma
\newmdtheoremenv[%
linecolor=lightgray!30,
backgroundcolor=lightgray!30,%
innertopmargin=3pt,]{lem}{Lemma}[section]

%riquadro definizione
\newmdtheoremenv[%
linecolor=c_def!50,
backgroundcolor=c_def!50,%
innertopmargin=3pt]{defn}{Definizione}[section]

%riquadro codice
\newmdtheoremenv[%
linecolor= c_alg!24,
backgroundcolor=c_alg!24,%
innertopmargin=3pt]{alg}{Algoritmo}[section]

\begin{document}
\maketitle

Qui sono raccolti tutti i contenuti di tutte le slide e dei vari materiali sparsi sul sito della 
prof M. Moscarini; a questi materiali sono aggiunti appunti, esempi ed esercizi mostrati a lezione
ed esami svolti.\\ 
I sorgenti LaTeX sono disponibili su GitHub 
\href{https://github.com/Halolegend94/uni_database_appunti}{\textcolor{blue}{QUI}}, dove troverete 
anche la versione più aggiornata di questo documento (main.pdf).\\\\

Ringrazio per l'aiuto nella correzione di errori e sviste nel testo ed esercizi: Federica Granese, 
Michael Aratari, Emanuele Alessi, Luca De Santis, Mirco Cassaro, Michele Laurenti, Franco Bertollini,
Matteo Cellucci, Andrea Coletta, Marco Esposito.



\newpage
\setcounter{tocdepth}{3}
\tableofcontents
\newpage
\section{Introduzione}
I sistemi di gestione di basi di dati (DBMS) sono strumenti software per la gestione
di grandi masse di dati. Prima dell'avvento dei database, ogni programma aveva il suo
file privato, organizzato \emph{sequenzialmente} e la gestione dei dati era affidata
al filesystem. Ci\`o causava problemi di:

\begin{itemize}
 \item \textbf{ridondanza}: diversi files venivano replicati, dovendo essere condivisi
 con pi\`u applicazioni;
 \item \textbf{inconsistenza}: se un'informazione veniva aggiornata, tale aggiornamento 
 poteva riguardare solo una copia del dato;
 \item \textbf{dipendenza dei dati}: i dati venivano strutturati dalle applicazioni, in
 base al loro utilizzo.
\end{itemize}

Per ovviare a questi problemi si inizi\`o a progettare le \textbf{basi di dati}, le quali
videro una grande svolta quando alcuni ingegneri dell'IBM introdussero negli anni '70 il
\emph{modello relazionale}.

  \subsection{Le basi di dati}
  Le caratteristiche di una base di dati sono:
  \begin{itemize}
   \item \textbf{multiuso e integrazione}: la stessa base di dati pu\`o essere utilizzata 
   da diverse applicazioni con diversi scopi;
   \item \textbf{indipendenza e controllo centralizzato}: i dati non sono gestiti dalle
   applicazioni ma da un software dedicato, il quale ne gestisce anche le regole di 
   accesso.
  \end{itemize}
 I vantaggi che ne derivano sono la minima ridondanza, integrità dei dati e sicurezza.
 
 \subsubsection{Integrità}
 I dati devono rispettare dei vincoli che esistono nella realtà di interesse. Consideriamo
 ad esempio il database di InfoStud della Sapienza:
 \begin{itemize}
  \item uno studente risiede in \emph{una sola} città (\textbf{dipendenze funzionali});
  \item la matricola \emph{identifica univocamente} uno studente (\textbf{vincoli di chiave});
  \item un voto è un intero positivo tra 18 e 30 (\textbf{vincoli di dominio}). 
 \end{itemize}
 
 \subsubsection{Sicurezza}
 I dati devono essere protetti da accessi non autorizzati. Il \textbf{DBA} (\emph{Database 
 Administrator}) deve considerare:
 \begin{itemize}
  \item valore corrente per \emph{accessi autorizzati};
  \item valore corrente per accessi \emph{non autorizzati};
  \item \emph{chi} può accedere \emph{a quali} dati e in \emph{quale modalità};
  \item definire regole di accesso ed effetti relativi a una violazione.
 \end{itemize}
 
 \subsection{Transazione}
 \begin{defn}Si definisce \textbf{transazione} una sequenza di operazioni che costituiscono 
 un'unica operazione logica.\end{defn}
 \begin{exmp}
  ``Trasferire €1000 da c/c1 al c/c2''
    \begin{enumerate}
     \item cerca c/c1;
     \item modifica $saldo$ in $saldo - 1000$;
     \item cerca c/c2;
     \item modifica $saldo$ in $saldo + 1000$;
    \end{enumerate}
 \end{exmp}
 Una transazione deve essere eseguita completamente (\emph{committed}) o non deve essere eseguita
 affatto (\emph{rolled back}).
 
 \subsection{Ripristino}
 Può capitare che, a causa di un malfunzionamento del sistema (ad esempio sbalzi di tensione), la
 base di dati si trovi con delle informazioni corrotte. In questi casi, per ripristinare il valore
 corretto dei dati, si hanno due possibili modi di agire:
 \begin{itemize}
  \item sfruttare il \textbf{transaction log}, che contiene i dettagli di tutte le transazioni,
  tra cui valori precedenti e successivi alla modifica;
  \item ripristinare l'ultimo \textbf{dump} effettuato. Il dump è una copia periodica del
  database.
 \end{itemize}
 
 \subsection{Compiti del DBA}
 Il Database Administrator ha vari compiti, tra i quali ricadono le scelte di progettazione
 della base di dati. Esse implicano la definizione di:
 \begin{itemize}
  \item schema logico
  \item schema fisico
  \item sottoschema o viste
 \end{itemize}
 inoltre al DBA spetta l'onere di mantenere il sistema.






\newpage
\input{sezioni/02-Modello-Relazionale.tex}
\newpage
\input{sezioni/03-Algebra-Relazionale.tex}
\newpage
\input{sezioni/04-Terza-Forma-Normale.tex}
\newpage
\input{sezioni/05-Organizzazione-Fisica.tex}
\newpage
\input{sezioni/06-Concorrenza.tex}
\newpage
\input{sezioni/07-Esami-Svolti.tex}
\end{document}
